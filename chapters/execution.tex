\chapter{Negotiation Execution}\label{chap:negotiation_execution}

The negotiation execution section outlines the questions, arguments, and refutations used during the negotiation process, the parties involved, and insights gained. This structure ensures a well-prepared and organized approach to the negotiation, increasing the likelihood of achieving the desired outcome.


\section{Prepared Questions, Arguments, and Refutations}

\begin{enumerate}
    \item \textbf{Question}: How do the current wireless communication technologies in the MOOD-Sense project align with the company's sustainability goals?\\
    \textbf{Argument}: Implementing a Thread mesh wireless network using a sustainable engineering approach can better align the project with the company's sustainability objectives.\\
    \textbf{Refutation}: Should stakeholders express concerns regarding the cost or complexity of the proposed solution, emphasize the long-term benefits of a sustainable approach, including reduced energy consumption, increased efficiency, and potential for scalability.
    \item \textbf{Question}: What challenges do you foresee in integrating existing devices and sensors into the new Thread mesh wireless network protocol?\\
    \textbf{Argument}: The multiprotocol support offered by the nRF devices, such as running BLE and Thread antennas concurrently, enables seamless integration without the need for device replacement.\\
    \textbf{Refutation}: Address concerns about compatibility or integration issues by providing detailed technical information and potential solutions for overcoming these challenges.
    \item \textbf{Question}: How does the proposed Thread mesh wireless network contribute to the overall effectiveness and efficiency of the MOOD-Sense project?\\
    \textbf{Argument}: The Thread mesh wireless network offers enhanced connectivity, interoperability, and communication among devices, leading to improved patient care and safety.\\
    \textbf{Refutation}: If stakeholders raise doubts about the benefits of the proposed network, provide data and examples from other successful implementations to demonstrate its effectiveness and positive impact.
    \item \textbf{Question}: How does the implementation of low-power strategies contribute to the sustainability of the MOOD-Sense project?\\
    \textbf{Argument}: Utilizing low-power strategies in the Thread mesh wireless network can significantly reduce energy consumption, making the project more environmentally friendly and cost-effective.\\
    \textbf{Refutation}: If the supervisor is concerned about the performance trade-offs, emphasize the balance between energy efficiency and reliable communication, as well as the potential for energy-saving techniques such as sleep modes and adaptive power management.
    \item \textbf{Question}: How can device availability and compatibility with the proposed Thread mesh wireless network be ensured?\\
    \textbf{Argument}: The chosen hardware components, such as nRF52840 DK, nRF52840 Dongle, and Raspberry Pi 4 B, are widely available and compatible with the Thread mesh network protocol, facilitating seamless integration.\\
    \textbf{Refutation}: Should device availability be a concern, discuss alternative options or backup plans to source the necessary components, ensuring minimal disruption to the project timeline.
    \item \textbf{Question}: How will the Thread mesh wireless network perform in various locations and environments within the MOOD-Sense project?\\
    \textbf{Argument}: The Thread mesh network is designed to provide reliable and robust communication in a range of environments, ensuring consistent performance across different locations and settings.\\
    \textbf{Refutation}: In case of concerns about network performance in specific environments, suggest conducting testing and optimization to guarantee reliable communication under all conditions.\\
\end{enumerate}


\section{Involved Parties}

The negotiation involved only the Company Supervisor. As the sole negotiation counterpart during the execution, their support was essential for successfully implementing the proposed solution.

\subsection*{Insights}

\begin{enumerate}
    \item Understanding the company supervisor's priorities, goals, and concerns helps to tailor arguments and responses during the negotiation better.
    \item Identifying potential areas of collaboration and synergy between the proposed solution and other subprojects within the MOOD-Sense initiative, leveraging these connections to build support for the proposal.
    \item Recognizing the importance of communication and relationship-building in achieving a successful negotiation outcome, establishing rapport, and fostering a collaborative atmosphere.
    \item Despite the initial plan to involve multiple stakeholders, the negotiation ultimately took place solely between the intern and the company supervisor. This allowed for a more focused and personalized discussion, which contributed to the intern's ability to sway the supervisor's opinion.
    \item During the negotiation execution process, specific arguments regarding the long-term benefits of the proposal, the alignment with sustainability goals, and the technical feasibility of the plan resonated with the company supervisor. By demonstrating a deep understanding of the project's needs and offering well-reasoned arguments, the intern effectively addressed potential concerns and ultimately gained the supervisor's support for the proposed solution.
\end{enumerate}
