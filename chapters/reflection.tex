\chapter{Negotiation Reflection}\label{chap:negotiation_reflection}

This section reflects on the negotiation process that took place between the student and the company supervisor, discussing the lessons learned, the effectiveness of the negotiation strategies employed, and potential improvements for future negotiations.


\section{Key Lessons Learned}

One of the most important lessons learned from this negotiation was the value of thorough preparation. Researching the subject matter, understanding the needs and concerns of the involved parties, and anticipating potential counterarguments provided a strong foundation for a successful negotiation. Active listening and clear, concise communication also significantly fostered constructive dialogue, allowing both parties to present their perspectives and reach a mutual understanding effectively.

The ability to adapt to unexpected changes, such as the shift from a multi-stakeholder negotiation to a one-on-one discussion, was essential in maintaining focus on the negotiation objectives and adjusting strategies accordingly. Additionally, establishing rapport and trust with the company supervisor facilitated a more open and collaborative atmosphere, ultimately contributing to a more favorable negotiation outcome.


\section{Areas for Improvement}

In future negotiations, it may be beneficial to involve a more diverse group of stakeholders, ensuring that a broader range of perspectives and interests are taken into account when discussing the proposal. Allocating sufficient time for the negotiation process, including adequate time for research, preparation, and the negotiation itself, can also enhance the overall effectiveness of the negotiation and lead to more informed decision-making.

Developing stronger conflict resolution skills can help address disagreements and potential roadblocks more effectively, creating a smoother path toward reaching a mutually beneficial agreement.


\section{Conclusion}

The negotiation between the student and the company supervisor provided valuable insights into the importance of preparation, communication, adaptability, and relationship-building in achieving a successful negotiation outcome. By reflecting on these experiences and identifying areas for improvement, future negotiations can be approached with increased confidence and skill, ultimately leading to more effective and sustainable project outcomes.
