\chapter{Research}\label{chap:research}

This section addresses the fundamental principles and best practices of sustainable engineering in designing and implementing a Thread mesh wireless network for the MOOD-Sense project. The research aims to provide a comprehensive understanding of sustainable engineering principles and how they can be applied to various aspects of the network.


\section{Key Principles and Best Practices of Sustainable Engineering}

Creating a sustainable Thread mesh wireless network in the MOOD-Sense project requires consideration of several key principles and best practices of sustainable engineering. These principles can guide the design and implementation process, ensuring that environmental impact, resource consumption, and the overall sustainability of the network are taken into account:

\begin{enumerate}
    \item \textbf{Life Cycle Thinking}: Assess the environmental impact of network components throughout their entire life cycle, from raw material extraction to end-of-life disposal.
    \item \textbf{Energy Efficiency}: Prioritize energy-efficient hardware components and software strategies, including low-power and energy-efficient communication protocols.
    \item \textbf{Waste Minimization}: Reduce waste generation during production, deployment, and maintenance by employing modular design principles and recyclable materials.
    \item \textbf{Durability and Reliability}: Ensure the network's longevity and reliability using high-quality materials, robust design practices, and fault-tolerant communication protocols.
    \item \textbf{Scalability and Flexibility}: Design the network to be adaptable to the evolving needs of the MOOD-Sense project, reducing the environmental impact associated with creating and disposing of new network components.
\end{enumerate}

By considering these key principles and best practices of sustainable engineering, the design and implementation of a Thread mesh wireless network within the MOOD-Sense project can be conducted to minimize environmental impact, optimize resource consumption, and promote responsible technological development \cite{tan2010sustainable}.


\section{Selecting and Optimizing Hardware and Software Components}
Selecting and optimizing hardware and software components of the Thread mesh network to minimize environmental impact and resource consumption while ensuring efficient communication within the MOOD-Sense framework involves the following key steps:

\begin{enumerate}
    \item \textbf{Hardware Selection}: Choose low-power hardware components, such as the nRF52840 DK and nRF52840 Dongle, sourced from ethical manufacturers with minimal environmental impact throughout their life cycle \cite{sabovic2021demonstration}.
    \item \textbf{Software Optimization}: Implement energy-saving software strategies and optimize communication protocols to ensure efficient and reliable performance of the hardware components within the MOOD-Sense framework \cite{s20174779}.
    \item \textbf{Modular and Scalable Design}: Develop a design approach that allows for easy replacement or upgrade of components, reducing waste generation and resource consumption during maintenance and system upgrades \cite{s20072028}.
    \item \textbf{Stakeholder Engagement}: Involve relevant stakeholders, such as healthcare professionals, patients, and caregivers, in the decision-making process to address their needs and preferences, contributing to the overall success of the project.
    \item \textbf{Lifecycle Management}: Employ sustainable engineering practices, such as component standardization and remote management solutions, to minimize resource consumption and environmental impact during production, deployment, and maintenance of the network.
\end{enumerate}

The Thread mesh wireless network can minimize environmental impact and resource consumption while maintaining efficient communication within the MOOD-Sense project by focusing on these five key aspects in the selection and optimization process.


\section{Implementing Low-Power Strategies}

Implementing low-power strategies in the Thread mesh wireless network protocol can contribute to the sustainability and overall effectiveness of the MOOD-Sense project by reducing energy consumption, prolonging battery life, and minimizing the environmental impact associated with energy production and consumption. Some low-power strategies that can be incorporated into the Thread mesh wireless network protocol include:

\begin{enumerate}
    \item \textbf{Power Management Modes}: Implement various power management modes, such as sleep, idle, and active states, that allow devices to conserve energy when not in use or during periods of low activity.
    \item \textbf{Adaptive Power Management}: Utilize adaptive power management techniques that automatically adjust the power consumption of devices based on their current operational requirements, such as adjusting the transmission power and frequency based on the proximity of neighboring devices.
    \item \textbf{Efficient Communication Protocols}: Optimize communication protocols to minimize overhead and reduce the time devices spend in high-power transmission and reception modes. This can be achieved by using data compression techniques, efficient routing algorithms, and optimized network topology.
    \item \textbf{Power-Aware Scheduling}: Implement power-aware scheduling algorithms that prioritize tasks based on their power consumption and time-sensitive nature, ensuring that energy-intensive tasks are executed during periods of optimal energy availability and minimizing the overall energy consumption of the network.
    \item \textbf{Energy Harvesting}: Explore the possibility of integrating energy harvesting techniques, such as solar or kinetic energy, to supplement the power supply of devices and reduce their reliance on traditional energy sources.
\end{enumerate}

By incorporating these low-power strategies into the Thread mesh wireless network protocol, the MOOD-Sense project can improve its sustainability and overall effectiveness by reducing energy consumption, prolonging the operational lifetime of devices, and minimizing the environmental impact associated with energy production and consumption \cite{Thread_Low_Power_2018}.


\section{Integrating Existing End Devices and Sensors}

Integrating existing end devices and sensors into the new Thread mesh wireless network protocol without replacement can be achieved by ensuring compatibility and minimizing environmental impact and resource consumption. Considering the multiprotocol support offered by the selected nRF devices, the following strategies can be employed to facilitate the integration of existing devices:

\begin{enumerate}
    \item \textbf{Multiprotocol Support}: Leverage the multiprotocol capabilities of the nRF devices, which can run both BLE and Thread antennas concurrently, to connect existing devices that do not have Thread support. This allows for seamless communication between the new Thread mesh network and the existing devices using BLE \cite{nordic_multiprotocol_support}.
    \item \textbf{Software and Firmware Updates}: Provide software and firmware updates for the existing end devices and sensors to ensure compatibility with the Thread mesh network protocol, enabling them to communicate effectively with the new network infrastructure.
    \item \textbf{Modular Adapters}: Design and develop modular adapters that can be attached to the existing end devices and sensors, enabling them to connect to the Thread mesh network without requiring extensive hardware modifications or replacements \cite{6916657}.
    \item \textbf{Interoperability Standards}: Ensure adherence to established interoperability standards and guidelines to facilitate seamless communication and data sharing among various devices within the MOOD-Sense framework, regardless of their communication protocols.
    \item \textbf{Resource Optimization}: Optimize the integration process to minimize the consumption of additional resources, such as energy and materials, while maintaining the functionality and effectiveness of the existing end devices and sensors within the MOOD-Sense project.
\end{enumerate}

By employing these strategies, existing end devices and sensors can be integrated into the new Thread mesh wireless network protocol without needing replacement, ensuring compatibility and minimizing environmental impact and resource consumption in the MOOD-Sense project.
