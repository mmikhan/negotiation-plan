\chapter{Rationale}\label{chap:rationale}


\section{Introduction}

Dementia is a progressive neurological disorder that presents significant challenges to patients, caregivers, and healthcare professionals. With the increasing global prevalence of dementia, there is a pressing need for innovative solutions to manage and mitigate its impact. The MOOD-Sense research project, based at the Centre of Applied Research Biobased Economy at Hanze University of Applied Sciences Groningen, addresses this issue by leveraging the potential of IoT devices to detect and predict challenging behavior in dementia patients. By developing an early warning system that combines sensor technology, artificial intelligence, and wireless communication, MOOD-Sense aims to provide real-time feedback for healthcare professionals, improving patient care and safety. Integrating sustainable engineering principles into the MOOD-Sense system offers an opportunity to take the project to the next level, minimizing environmental impact, optimizing resource consumption, and creating a more reliable, efficient solution for addressing the challenges posed by dementia \cite{mood-sense_2021}.


\section{Present Situation}

The initial plan was to utilize three wireless communication technologies, BLE, ZigBee, and Wi-Fi, for network communication. Despite this intention, active network protocol implementation has yet to occur. Various subprojects related to the MOOD-Sense framework, such as dementia patient behavior registration and environmental context monitoring, have been conducted concurrently. The absence of a central network communication protocol has led to the fragmentation of devices across subprojects, impeding data sharing and integration. The proposal of employing a Thread mesh wireless network has been introduced to tackle this issue, given its mesh structure, affordability, and reliability capabilities. This network protocol would facilitate the connection of BLE, ZigBee, and Wi-Fi in a centralized manner, enabling seamless connectivity, interoperability, and communication among all devices within the MOOD-Sense framework.


\section{Desired Situation}

The desired outcomes of the ongoing research involve the establishment of a Thread mesh wireless network protocol, with the primary objective of facilitating seamless communication among devices within the MOOD-Sense framework. By successfully implementing a Thread mesh network, interoperability, and data sharing among various subprojects will be significantly improved, ultimately contributing to enhanced patient care and safety. Furthermore, integrating sustainability principles in the development and deployment of this network protocol can reduce environmental impact and optimize resource consumption, aligning the project with the goals of responsible and efficient technological advancement. In this desired scenario, the MOOD-Sense project will not only address the challenges posed by dementia through innovative IoT solutions but also demonstrate a commitment to sustainable engineering practices, setting a precedent for future research and development initiatives in the field.


\section{Research Questions}

\subsection*{Main Research Question}

How can a Thread mesh wireless network be designed and implemented within the MOOD-Sense project using a sustainable engineering approach?

\subsection*{Sub-Research Questions}

\begin{enumerate}
    \item What are the fundamental principles and best practices of sustainable engineering relevant to designing and implementing a Thread mesh wireless network in the MOOD-Sense project?
    \item How can the hardware and software components of the Thread mesh network be selected and optimized to minimize environmental impact and resource consumption while ensuring efficient and reliable communication within the MOOD-Sense framework?
    \item How can the implementing low-power strategies in the Thread mesh wireless network protocol contribute to the sustainability and overall effectiveness of the MOOD-Sense project?
    \item How can existing end devices and sensors be integrated into the new Thread mesh wireless network protocol without replacement, ensuring compatibility and minimizing environmental impact and resource consumption?
\end{enumerate}
