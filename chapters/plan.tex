\chapter{Negotiation Plan}\label{chap:negotiation_plan}

This section outlines a negotiation plan for discussing the proposal to implement a sustainable Thread mesh wireless network within the MOOD-Sense project with the company supervisor and other stakeholders. The plan aims to address potential objections, identify mutual needs, and develop a consensus strategy.


\section{Establishing the Framework}

This section introduces the context and objectives of the negotiation and the needs of the involved parties.


\subsection*{Personal  Needs}

\begin{enumerate}
    \item Obtain approval and support for the proposal to integrate the sustainable Thread mesh wireless network into the MOOD-Sense project.
    \item Gather necessary resources, including funding, equipment, and technical expertise, to implement the proposed solution successfully.
\end{enumerate}

\subsection*{Organizational Needs}

\begin{enumerate}
    \item Ensure the proposed solution aligns with the company's goals and objectives, particularly regarding patient care, safety, and sustainability.
    \item Evaluate the proposed solution's feasibility, cost-effectiveness, and potential return on investment.
\end{enumerate}


\section{Nature of the Interaction}

This section provides an overview of the negotiation format, the parties involved, and the potential dynamics of the interaction.

\subsection*{Format}
The negotiation will be primarily held between the individual and the company supervisor, focusing on the proposed Thread mesh wireless network. Other stakeholders, such as technical team members, healthcare professionals, and representatives of the end device and sensor manufacturers, may be included in the discussion to provide valuable insights and address specific concerns.

\subsection*{Situational Parameters}

\begin{enumerate}
    \item The negotiation will be conducted formally, preferably in a meeting room, to maintain a professional atmosphere.
    \item The duration of the negotiation should be limited to ensure focus and efficiency.
    \item If multiple negotiation plans are under consideration, it is essential to emphasize that each proposal should be evaluated independently and based on its merits as long as they align with the project's goals.
\end{enumerate}

\subsection*{Stakeholders}

\begin{enumerate}
    \item \textbf{Company Supervisor}: Responsible for overseeing the MOOD-Sense project and approving major decisions, including the proposed Thread mesh wireless network.
    \item \textbf{Technical Team Members}: Responsible for the development and implementation of the proposed solution.
    \item \textbf{Healthcare Professionals}: Users of the MOOD-Sense system who can provide valuable input regarding its effectiveness and applicability in a clinical setting.
    \item \textbf{Representatives of End Device and Sensor Manufacturers}: Provide technical information and support for the integration of existing devices and sensors.
\end{enumerate}

\subsection*{Goals}

\begin{enumerate}
    \item Achieve a mutual understanding of the proposal, its benefits for the MOOD-Sense project, and the company's broader objectives.
    \item Address potential objections and concerns raised by stakeholders.
    \item Reach a consensus on the implementation of the proposed solution.
\end{enumerate}

\subsection*{Strategy}

\begin{enumerate}
    \item Begin by presenting a clear and concise overview of the proposed solution, highlighting its potential benefits in terms of sustainability, communication efficiency, and patient care.
    \item Address potential objections by providing evidence-based counterarguments, demonstrating the feasibility and cost-effectiveness of the proposed solution.
    \item Engage in active listening and ask open-ended questions to understand stakeholder concerns better and gather valuable feedback.
    \item If applicable, consider intercultural issues by researching cultural norms and expectations and adjusting communication style accordingly.
    \item Maintain a collaborative approach throughout the negotiation, emphasizing the shared goals of improving patient care and advancing the MOOD-Sense project.
\end{enumerate}

By following this negotiation plan, the discussion surrounding the implementation of a sustainable Thread mesh wireless network within the MOOD-Sense project can be productive and result in a mutually beneficial outcome.


\section{Formulating a Strategic Approach}

This section aims to outline the components of the negotiation situation, identify target points and resistance points, determine relative power, assess the other party's knowledge, and explore strategies for building rapport. As a master's student intern within the MOOD-Sense project, it is essential to navigate the complexities of the large research initiative and consider the potential challenges associated with various convincing parties of the merits of sustainable wireless network development.

\subsection*{Components of the Negotiation Situation}

\begin{enumerate}
    \item MOOD-Sense is a large research initiative with multiple subprojects running concurrently.
    \item Communication and coordination among different projects may be challenging.
    \item The intern's position within the project is not inherently powerful, but persuasive arguments can still be made.
\end{enumerate}

\subsection*{Target Points}

\begin{enumerate}
    \item Obtain approval for implementing the sustainable Thread mesh wireless network within the MOOD-Sense project.
    \item Acquire necessary resources, including funding, equipment, and technical expertise.
    \item Garner support from key stakeholders, including the company supervisor, technical team members, and healthcare professionals.
\end{enumerate}

\subsection*{Resistance Points}

\begin{enumerate}
    \item Difficulty in convincing stakeholders of the value of sustainable wireless network development.
    \item Limited resources or competing priorities within the MOOD-Sense project.
    \item Hesitation from stakeholders due to the intern's position within the project.
\end{enumerate}

\subsection*{Relative Power}

\begin{enumerate}
    \item As a master's student intern, the relative power is limited compared to the company supervisor and other stakeholders.
    \item However, the power of persuasion and the strength of the proposal can make a significant impact.
\end{enumerate}

\subsection*{Other Party's Knowledge}

\begin{enumerate}
   \item Research the backgrounds, interests, and priorities of key stakeholders.
   \item Identify potential allies who share similar goals and values.
   \item Understand the company's overall goals and objectives, as well as the specific needs and challenges of the MOOD-Sense project.
\end{enumerate}

\subsection*{Building Rapport}

\begin{enumerate}
    \item Engage in active listening and ask open-ended questions to understand stakeholder concerns better and gather valuable feedback.
    \item Demonstrate empathy and understanding of the challenges faced by different parties.
    \item Showcase personal expertise and commitment to the project, emphasizing the proposal's benefits for the MOOD-Sense project and the company.
    \item Establish common ground with stakeholders by highlighting shared goals and values.
    \item Maintain a positive and respectful attitude throughout the negotiation process.
\end{enumerate}

\subsection*{BATNA (Best Alternative to a Negotiated Agreement)}

\begin{enumerate}
    \item Continue researching and refining the proposal, seeking external expertise or resources if necessary.
    \item Identify alternative solutions that may be more easily integrated within the MOOD-Sense project or have a stronger appeal to stakeholders.
    \item Seek opportunities to collaborate with other subprojects to create a more comprehensive and coordinated approach.
\end{enumerate}

By establishing a collaborative framework and building rapport with stakeholders, the negotiation process can be more productive and create a supportive environment for implementing the sustainable Thread mesh wireless network within the MOOD-Sense project.
